\documentclass{article}
\title{New results for the Degree/Diameter Problem}
\author{Yawara ISHIDA}

\usepackage{bm}
\usepackage{amsthm}
\usepackage{amsmath}
\usepackage{amsfonts}
\usepackage{hyperref}

\newtheorem{Def}{Definition}
\newtheorem{Lem}{Lemma}
\newtheorem{Cor}[Lem]{Corollary}
\newtheorem*{Th}{Theorem}

\newcommand{\Z}{\mathbb Z}
\newcommand{\N}{\mathbb N}
\newcommand{\B}{\mathrm{B}}

\begin{document}
\maketitle
\begin{abstract}
The Degree/Diameter Problem is one of the most famous problem in graph theory. 
We consider the problem for the case of diameter 2.
A lower bound of the order of $(d,2)$-graph is known as the Brown's construction.
In this paper, we generalized his construction.
Then we found two new records of degree $306$ and $307$ in the graphs generated by our new construction.
One is $(306,2)$-graph with $88732$ vertices.
The other is $(307,2)$-graph with $88733$ vertices.
\end{abstract}

\section{Introduction}
A {\it graph} $G=(V,E)$ consists of a set $V$ called {\it vertices} and a set $E \subset \{(v,w) \in V^2 | v \neq w \}$ called {\it edges}.
If $(v,w)$ is in $E$, it is said that $v$ and $w$ are {\it adjacent}, which is denoted by $v \sim w$.
The {\it order} $|G|$ of the graph is the size of the set of vertices. 
The {\it neighbors} $N(v)$ of the vertex $v$ is a set of vertices which are adjacent to $v$.
The {\it degree} of the vertex $\delta(v)$ is a size of neighbors $| N(v) |$.  
The {\it degree} of the graph $\Delta(G)$ is the maximal degree  of the vertex.
\iffalse
The graph is {\it regular} if every vertex's degree are same.
\fi
The {\it distance} for each pair $(v,w)$ of vertices is the shortest path length between $v$ and $w$. 
The {\it diameter} $D(G)$ of the graph is the  maximum distance of all pairs of vertices. 

The {\it Degree/Diameter problem} is the problem of finding the largest possible number of vertices in graphs of given degree $\Delta$ and diameter $D$~\cite{MilSir2005}. 
The order of the graph of degree $\Delta$ ($\Delta > 2$) and diameter $D$ is easily seen to be bounded by 
\[ 1 + \Delta \sum_{k=1}^{D-1} (\Delta - 1)^k\]
which is called {\it Moore bound}.
Moore bound is a general upper bound. 
On the other hand, already known lower bound of small degree and small diameter are availabe at \url{http://combinatoricswiki.org}. 
Especially for case of $D=2$ and large degree there exists the general construction called the {\it Brown's construction}~\cite{MilSir2005}.
Given the finite field $F_q$ where q is a power of prime, we construct the graph $\B(F_q)$ whose vertices are lines in $F_q^3$ and two lines are adjacent if and only if they are orthogonal. 
We call it the Brown's graph.
The order of $B(F_q)$ is $q^2+q+1$ and the degree of it is $q+1$. The diameter of it is $2$ because $B(F_q)$ includes many triangles. 
$F_q$ is isotropic so any lines are symmetric in $F_q$, the degree of the vertex of $\B(F_q)$ is $q+1$ or $q$. 
There exists $q+1$ vertices of degree $q$ in $\B(F_q)$.
If $q$ is a power of $2$, there exists $(q+1,2)$-graph with $q^2+q+2$ vertices by modifying $\B(F_q)$~\cite{journals/networks/ErdosFH80}. 
In this paper, we generalize the Brown's construction, in which we replace a field with a ring, and search new records of the Degree/Diameter Problem.

%%% Hmm %%%
\iffalse

The {\it tensor product} of two graphs $G_1=(V_1,E_1),G_2=(V_2,E_2)$ is a graph $G_1 \otimes G_2$ such that the vertex set is a cartesian product of $V_1$ and $V_2$ and two vertices $(v_1,v_2) \sim (w_1,w_2)$ if and only if $v_1 \sim v_2$ and $w_1 \sim w_2$. If $G_1, G_2$ are regular, the tensor product $G_1 \times G_2$ is also regular. 

\fi
%%% --- %%%

Let $R$ be a ring with unity. 
$R^*$ denotes the set of invertible elements of $R$.
$R^3$ is naturally seen as $R$-module. 
The addition and $R$-action are defined by coordinate-wise.
The {\it inner product} $\cdot: R^3 \times R^3 \Rightarrow R$ is defined as follows
\[ (v_1,v_2,v_3) \cdot (w_1,w_2,w_3) = v_1 w_1 + v_2 w_2 + v_3 w_3 \]
${\bm v}$ and ${\bm w}$ are {\it orthogonal} if and only if the inner product vanishes, namely ${\bm v} \cdot {\bm w} = 0$.
The {\it cross product} $\times: R^3 \times R^3 \Rightarrow R$ is defined as follows
\[ (v_1,v_2,v_3) \cdot (w_1,w_2,w_3) = ( v_2 w_3 - v_3 w_2, v_3 w_1 - v_1 w_3, v_1 w_2 - v_2 w_1 ) \]
A {\it domain} $D$ is a ring without zero divisors.
A {\it Euclidean domain} is a domain $E$ with a function called degree $d: E \setminus \{0\} \Rightarrow \N$ such that for all non-zero $a,b$ in $E$ there exists $q,r \in E, a = q b + r$ where $d(r) < d(b)$. Every {\it Euclidean domain} is a {\it unique factorization domain}, in which for all $r$ in $E$, there exist prime elements $u_i$ and natural numbers $k_i$ such that $r = \Pi_i u_i^{k_i}$.
The ring of integers $\Z$ is a example of the Euclidean domains whose degree function is an identity function.


\section{Generalized Brown's Construction}
\begin{Def}
Let $R$ be a ring with unit. The vertex set $V$ of the generalized Brown's graph $\B(R)$ is \[ V = ( R^3 \setminus \{\bm v | \exists r \in R, r \cdot {\bm v} = {\bm 0} \} ) / \sim\]
where $\bm v \sim \bm w$ if and only if there exists $k \in R^*$ such that $k \cdot {\bm v} = {\bm w}$. The two vertices $[\bm v]$ and $[\bm w]$ are adjacent if and only if ${\bm v} \cdot {\bm w} = 0$.
\end{Def}

The adjacency of the definition above is well-defined because the orthogonality does not depends on the selection of representitives. We call the above construction which gives a graph from a ring the {\it generalized Brown's construction}. It is clear that the new construction coincides old one when the ring is a field. 

%%% Hmmm %%%
\iffalse

\begin{Lem}
Let $R_1, R_2$ be rings with unit. 
\[ \B(R_1 \times R_2) \simeq \B(R_1) \otimes \B(R_2) \]
\end{Lem}

\begin{Cor}
The followig equations hold.
\begin{enumerate}
\item $ | \B(R_1 \times R_2) | = | \B(R_1) | \times | \B(R_2) | $
\item $ \Delta(\B(R_1 \times R_2)) = \Delta(\B(R_1)) \times \Delta( \B(R_2) )$
\end{enumerate}
\end{Cor}

\fi 
%%% --- %%%

\begin{Lem}\label{Lem:regular}
Let $E$ be a Euclidean domain and $u$ be a prime element in $E$. 
If $E/(u^k)$ is a finite ring, then the degree of the vertex of $\B(E/(u^k))$ is $\Delta$ or $\Delta-1$, 
where $(u^k)$ is the principal ideal generated by $u^k$. 
\end{Lem}

\begin{proof}
It is clear that the degrees of vertices represented by $(1,0,0), (0,1,0), (0,0,1)$ are same.
Let ${\bm v} = ([a],[b],[c])$ be a representitive of any vertex where $a,b,c \in E$. 
If any element of $[a],[b],[c]$ is not invertible in $E/(u^k)$, 
there exists natural numbers $l,m,n < k$ and some elements $a',b',c'$ in $E$ such that $a=u^l a', b=u^m b', c=u^n c'$. ${\bm v}$ is not a representitive of vertices because of $[u^{min(l,m,n)}] \cdot {\bm v} = ([0],[0],[0])$. 
Therefore, one element of $[a],[b],[c]$ is at leatst invertible.
If $[a]$ is invertible, there exists one-to-one correspondence $\overline{U}: N([(1,0,0)])  \rightarrow N([{\bm v}])$ such that for all $[{\bm w}] \in N([1,0,0])$, $\overline{U}([\bm w]) = [ {}^t\!U^{-1} {\bm w} ]$ where U is an invertible matrix defined as follows
\[
 U = \left(
 \begin{matrix}
  [a] & 0 & 0 \\
  [b] & 1 & 0 \\
  [c] & 0 & 1
 \end{matrix} \right)
\]
Therefore, $\delta([\bm v]) = \delta([(1,0,0)])$. In the same way, if $[b]/[c]$ is invertible, then $\delta([\bm v]) = \delta([(0,1,0)])/\delta([(0,0,1)])$. 

\end{proof}

\begin{Lem}\label{Lem:diameter}
Let $E$ be a Euclidean domain and $I$ be an ideal of $E$. The diameter of $\B(E/I)$ is 2.
\end{Lem}

\begin{proof}
For any two distinct vertices represented by $\bm v = ([v_1],[v_2],[v_3])$ and $\bm w=([w_1],[w_2],[w_3])$, consider the cross product $\bm v \times \bm w$. 
If $\bm v \times \bm w = \bm 0$, then $[v_i] \cdot \bm w = [w_i] \cdot \bm v$ for $i=1,2,3$. 
There exists $e \in E$ such that $I = (e)$ because any Euclidean domain is a princpal ideal domain.
Let $d$ be the greatest common divisor of $v_1$ and $v_2$, $v_3$, $e$. If $d$ is not a unity, there exists $e' \neq 1$ in $E$ such that $e=de'$. $\bm v$ is not a representitive because $[e'] \cdot \bm v = \bm 0$. Therefore $d$ is a unity, namely $v_1$ and $v_2$, $v_3$, $e$ are coprime.
Then there exist $a, b, c, d \in E$ such that $ a v_1 + b v_2 + c v_3 + d e = 1$ in $E$.
Seeing this formula in $E/I$, we get $[a] [v_1] + [b] [v_2] + [c] [v_3] = [1]$.
\[ \bm v = [1] \cdot \bm v = ( [a] [v_1] + [b] [v_2] + [c] [v_3] ) \bm v = ( [a] [w_1] + [b] [w_2] + [c] [w_3] ) \bm w\]
means $[\bm v] = [\bm w]$, which is a contradiction then $\bm v \times \bm w \neq \bm 0$. 
If $\bm v \times \bm w$ is a representitive of vertex, $[\bm v \times \bm w]$ is adjacent to $[\bm v]$ and $[\bm w]$.
If $\bm v \times \bm w = ([k_1],[k_2],[k_3])$ is not a representitive of vertex, there exist $[d]$ in $E/I$, where $d$ is a greatest common divisor of $k_1$, $k_2$ and $k_3$, and $\bm u$ in $(E/I)^3$ such that $\bm v \times \bm w = [d] \cdot \bm u$ and $\bm u$ is a representitive of vertex. $[\bm u]$ is adjacent to $[\bm v]$ and $[\bm w]$.
\end{proof}

\begin{Th}
The following equations hold.
\begin{enumerate}
\item $ |\B(\Z_{p^k})| = p^{2k}+p^{2k-1}+p^{2k-2} $
\item $ \Delta(\B(\Z_{p^k})) = p^k + p^{k-1} $
\item $ D(\Z_{p^k}) = 2$
\end{enumerate}
\end{Th}

\begin{proof}
It is straightforward to show the formula of the order of $\B(\Z_{p^k})$.

\begin{eqnarray*}
|\B(\Z_{p^k})| & = & \frac{|\Z_{p^k}|^3 - |\{ mp | 0 \leq m < k \}|^3}{|\Z_{p^k}|-|\{ mp | 0 \leq m < k \}|} \\ 
& = & \frac{(p^k)^3 - (p^{k-1})^3}{p^k-p^{k-1}} = p^{2k}+p^{2k-1}+p^{2k-2}
\end{eqnarray*}

It is only enough to show that the degree of the vertex represented by $(1,0,0)$ satisfy the formula of the degree of $\B(\Z_{p^k})$. 
It is clear that $\Z_p^{k}$ satisfies the assumption of Lemma \ref{Lem:regular}, then $\B(\Z_p^{k})$ is a regular graph. 
\begin{eqnarray*}
\Delta(\B(\Z_p^k)) & = & \delta([(1,0,0)]) = \frac{|\Z_{p^k}|^2 - |\{ mp | 0 \leq m < k \}|^2 }{|\Z_{p^k}|-|\{ mp | 0 \leq m < k \}|} \\
& = & \frac{(p^k)^2 - (p^{k-1})^2}{p^k-p^{k-1}} = p^k + p^{k-1}
\end{eqnarray*}

Applying Lemma \ref{Lem:diameter}, we get $D(\B(\Z_p^k)) = 2$

\end{proof}

We search new records among graphs by generalized Brown's construction.


\section{Acknowledgement}
Thank you, Ryosuke Mizuno, Nobuhito Tamaki, Sakie Suzuki !!!

\bibliographystyle{plain}
\bibliography{ref}

\end{document}