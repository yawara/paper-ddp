\documentclass{article}
\title{New results for the Degree/Diameter Problem}
\author{Yawara ISHIDA}

\usepackage{bm}
\usepackage{amsthm}
\usepackage{amsfonts}

\newtheorem{Def}{Definition}
\newtheorem{Lem}{Lemma}
\newtheorem{Cor}[Lem]{Corollary}
\newtheorem*{Th}{Theorem}

\newcommand{\Z}{\mathbb Z}
\newcommand{\N}{\mathbb N}
\newcommand{\EB}{\mathrm{EB}}

\begin{document}
\maketitle
\begin{abstract}
The Degree/Diameter Problem is one of the most famous problem in graph theory. 
We consider the problem for the case of diameter 2.
We extended the Brown's construction.
We found a new graph by $(306,2)$-graph with $88723$ vertices and $(307,2)$-graph with $88724$ by our construction.
\end{abstract}

\section{Introduction}
A {\it graph} $G=(V,E)$ consists of a set $V$ called {\it vertices} and a set $E \subset V^2$ called {\it edges}.
If $(v,w)$ is in $E$, it is said that $v$ and $w$ are {\it adjacent}, which is denoted by $v \sim w$.
If the vertex $v$ is adjacent to itself, the edge $(v,v)$ is called {\it loop}.
The graph $G$ without any loops is {\it simple}.
The {\it order} $|G|$ of the graph is the size of the set of vertices. 
The {\it neighbors} $N(v)$ of the vertex $v$ is a set of vertices which are adjacent to $v$.
The {\it degree} of the vertex $\delta(v)$ is a size of neighbors $| N(v) |$.  
The {\it degree} of the graph $\Delta(G)$ is the maximal degree  of the vertex.
The graph is {\it regular} if every vertex's degree are same.
The {\it distance} for each pair $(v,w)$ of vertices is the shortest path length between $v$ and $w$. 
The {\it diameter} $D(G)$ of the graph is the  maximum distance of all pairs of vertices. 
The {\it Degree/Diameter problem} is the problem of finding the graph with the maximum vertices for given degree $\Delta$ and diameter $D$. 
The order of a graph with degree $\Delta$ ($\Delta > 2$) of diameter D is easily seen to be bounded by 
\[ 1 + \Delta \sum_{k=1}^{D-1} (\Delta - 1)^k\]
which is called {\it Moore bound}.

The genaral constructions for small diameters are known. Especially for case of  $D=2$ there exists the general construction called the {\it Brown's construction}.
Given the finite field $F_q$ where q is a power of prime, we construct the graph $B(F_q)$ whose vertices are lines in $F_q^3$ and two lines are adjacent if and only if they are orthogonal. 
We call it the Brown's graph.
The order of $B(F_q)$ is $q^2+q+1$ and the degree of it is $q+1$. The diameter of it is 2 because $B(F_q)$ includes many triangles. 
Any lines are symmetric in $F_q$, so $B(F_q)$ is regular. However it is not simple because of including some loops. 
Removing any loops from $B(F_q)$, we get the simple graph whose degree of vertices are $q+1$ or $q$. 

%%% Hmm %%%
\iffalse

The {\it tensor product} of two graphs $G_1=(V_1,E_1),G_2=(V_2,E_2)$ is a graph $G_1 \otimes G_2$ such that the vertex set is a cartesian product of $V_1$ and $V_2$ and two vertices $(v_1,v_2) \sim (w_1,w_2)$ if and only if $v_1 \sim v_2$ and $w_1 \sim w_2$. If $G_1, G_2$ are regular, the tensor product $G_1 \times G_2$ is also regular. 

\fi
%%% --- %%%

Let $R$ be a ring with unity. 
$R^*$ denotes the set of invertible elements of $R$.
$R^3$ is naturally seen as R-module. 
The addition and $R$-action are defined by coordinate-wise.
The {\it inner product} $\cdot: R^3 \times R^3 \Rightarrow R$ is defined as follows
\[ (v_1,v_2,v_3) \cdot (w_1,w_2,w_3) = v_1 w_1 + v_2 w_2 + v_3 w_3 \]
${\bm v}, {\bm w}$ are {\it orthogonal} if and only if the inner product vanishes, namely ${\bm v} \cdot {\bm w} = 0$.
The {\it cross product} $\times: R^3 \times R^3 \Rightarrow R$ is defined as follows
\[ (v_1,v_2,v_3) \cdot (w_1,w_2,w_3) = ( v_2 w_3 - v_3 w_2, v_3 w_1 - v_1 w_3, v_1 w_2 - v_2 w_1 ) \]
A {\it domain} $D$ is a ring without zero divisors.
A {\it Euclidean domain} is a domain $E$ with a function called degree $d: E \setminus \{0\} \Rightarrow \N$ such that for all non-zero $a,b \in E$ there exists $q,r \in E, a = q b + r$ where $d(r) < d(b)$.
The ring of integers $\Z$ is a example of the Euclidean domains.


\section{Extended Brown's Construction}
\begin{Def}
Let $(R,+,0,*,1)$ be a ring with unit. The vertex set $V$ of the extended Brown's graph $\EB(R)$ is \[ V = ( R^3 \setminus \{\bm v | \exists r \in R, r \cdot {\bm v} = {\bm 0} \} ) / \sim\]
where $\bm v \sim \bm w$ if and only if $\exists k \in R^*, k \cdot {\bm v} = {\bm w}$. The two vertices $[\bm v],[\bm w]$ are adjacent if and only if ${\bm v} \cdot {\bm w} = 0$.
\end{Def}

The adjacency of the definition above is well-defined because the orthogonality does not depends on the selection of representitives. We call the above construction whish gives a graph from a ring the {\it extended Brown's construction}. It is clear that the new construction coincides old one when the ring is a field.

%%% Hmmm %%%
\iffalse

\begin{Lem}
Let $R_1, R_2$ be rings with unit. 
\[ \EB(R_1 \times R_2) \simeq \EB(R_1) \otimes \EB(R_2) \]
\end{Lem}

\begin{Cor}
The followig equations hold.
\begin{enumerate}
\item $ | \EB(R_1 \times R_2) | = | \EB(R_1) | \times | \EB(R_2) | $
\item $ \Delta(\EB(R_1 \times R_2)) = \Delta(\EB(R_1)) \times \Delta( \EB(R_2) )$
\end{enumerate}
\end{Cor}

\fi 
%%% --- %%%

\begin{Lem}
Let $E$ be a Euclidean domain and $u$ be a prime element in $E$. If $E/(u^k)$ is a finite ring, then $\EB(E/(u^k))$ is a regular graph, where $(u^k)$ is a principal ideal generated by $u^k$. Note that $\EB(E/(u^k))$ is not simple.
\end{Lem}

\begin{proof}
It is clear that the degrees of vertices represented by $(1,0,0), (0,1,0), (0,0,1)$ are same.
Let ${\bm v} = ([a],[b],[c])$ be a representitive of any vertex where $a,b,c \in E$. 
If any element of $[a],[b],[c]$ is not invertible in $E/(u^k)$, 
there exists natural numbers $l,m,n < k$ and some elements $a',b',c'$ in $E$ such that $a=u^l a', b=u^m b', c=u^n c'$. ${\bm v}$ is not a representitive of vertices because of $[u^{min(l,m,n)}] \cdot {\bm v} = ([0],[0],[0])$. 
Therefore, one element of $[a],[b],[c]$ is at leatst invertible.
If $[a]$ is invertible, there exists one-to-one correspondence $U^{-1}: N([v]) \rightarrow N([1,0,0])$ such that 
\end{proof}

\begin{Lem}
Let $E$ be a Euclidean domain and $I$ be an ideal of $E$. The diameter of $\EB(E/I)$ is 2.
\end{Lem}

\begin{proof}
For any two distinct vertices $[\bm v]$ and $[\bm w]$, consider the cross product $\bm v \times \bm w$. If $\bm v \times \bm w = \bm 0$, $v_i \cdot \bm w = w_i \cdot \bm v$ for $i=1,2,3$. 
For any vertex $[\bm v]$, the cordinate triple $(v_1,v_2,v_3)$ of ${\bm v}$ are coprime, then there exists $a,b,c \in E/I$ such that $a v_1 + b v_2+ c v_3 = 1$.
\[ \bm v = 1 \cdot \bm v = ( a v_1 + b v_2+ c v_3 ) \bm v = ( a w_1 + b w_2 + c w_3 ) \bm w\]
It is a contradiction then $\bm v \times \bm w \neq \bm 0$. 
If $\bm v \times \bm w$ is a representitive of vertex, $[\bm v \times \bm w]$ is adjacent to $[\bm v]$ and $[\bm w]$.
If $\bm v \times \bm w$ is not a representitive of vertex, there exist $k \in E/I$ and $\bm u \in (E/I)^3$ such that $\bm v \times \bm w = k \cdot \bm u$ and $\bm u$ is a representitive of vertex. $[\bm u]$ is adjacent to $[\bm v]$ and $[\bm w]$.
\end{proof}

\begin{Th}
The following equations hold.
\begin{enumerate}
\item $ |\EB(\Z_{p^k})| = p^{2k}+p^{2k-1}+p^{2k-2} $
\item $ \Delta(\EB(\Z_{p^k})) = p^k + p^{k-1} $
\item $ D(\Z_{p^k}) = 2$
\end{enumerate}
\end{Th}

\begin{proof}
It is straightforward to show the formula of the order of $\EB(\Z_{p^k})$.

\begin{eqnarray*}
|\EB(\Z_{p^k})| & = & \frac{|\Z_{p^k}|^3 - |\{ mp | 0 \leq m < k \}|^3}{|\Z_{p^k}|-|\{ mp | 0 \leq m < k \}|} \\ 
& = & \frac{(p^k)^3 - (p^{k-1})^3}{p^k-p^{k-1}} = p^{2k}+p^{2k-1}+p^{2k-2}
\end{eqnarray*}

It is straightforward to show the formula of the degree of the vertex represented by $(1,0,0)$
\begin{eqnarray*}
\delta([(1,0,0)]) & = & \frac{|\Z_{p^k}|^2 - |\{ mp | 0 \leq m < k \}|^2 }{|\Z_{p^k}|-|\{ mp | 0 \leq m < k \}|} \\
& = & \frac{(p^k)^2 - (p^{k-1})^2}{p^k-p^{k-1}} = p^k + p^{k-1}
\end{eqnarray*}
In the same way, 
\[\delta([(1,0,0)])=\delta([(0,1,0)])=\delta([(0,0,1)]) = p^k + p^{k-1}\]
holds.
The degree of any other vertex is same to the degree of the vertex represented by $(1,0,0)$. 
For all a representitive ${\bm v} = (v_1,v_2,v_3)$ of the vertex, the triple $(v_1,v_2,v_3)$ includes one invertible element at least. If $v_1$ is invertible element, 

\end{proof}

We search new graphs 

\begin{Th}
\end{Th}

\section{Acknowledgement}
Thank you!!!

\end{document}