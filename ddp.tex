\documentclass{article}
\title{New results for the Degree/Diameter Problem}
\author{Yawara ISHIDA}

\usepackage{bm}
\usepackage{amsthm}
\usepackage{amsmath}
\usepackage{amsfonts}
\usepackage{hyperref}

\newtheorem{Def}{Definition}
\newtheorem{Lem}{Lemma}
\newtheorem{Cor}[Lem]{Corollary}
\newtheorem*{Th}{Theorem}

\newcommand{\Z}{\mathbb Z}
\newcommand{\N}{\mathbb N}
\newcommand{\B}{\mathrm{B}}

\begin{document}
\maketitle
\begin{abstract}
The Degree/Diameter Problem is one of the most famous problem in graph theory. 
We consider the problem for the case of diameter 2.
We generalized the Brown's construction.
We found new graphs $(306,2)$-graph with $88723$ vertices and $(307,2)$-graph with $88724$ by our construction.
\end{abstract}

\section{Introduction}
A {\it graph} $G=(V,E)$ consists of a set $V$ called {\it vertices} and a set $E \subset V^2$ called {\it edges}.
If $(v,w)$ is in $E$, it is said that $v$ and $w$ are {\it adjacent}, which is denoted by $v \sim w$.
If the vertex $v$ is adjacent to itself, the edge $(v,v)$ is called {\it loop}.
The graph $G$ without any loops is {\it simple}.
The {\it order} $|G|$ of the graph is the size of the set of vertices. 
The {\it neighbors} $N(v)$ of the vertex $v$ is a set of vertices which are adjacent to $v$.
The {\it degree} of the vertex $\delta(v)$ is a size of neighbors $| N(v) |$.  
The {\it degree} of the graph $\Delta(G)$ is the maximal degree  of the vertex.
The graph is {\it regular} if every vertex's degree are same.
The {\it distance} for each pair $(v,w)$ of vertices is the shortest path length between $v$ and $w$. 
The {\it diameter} $D(G)$ of the graph is the  maximum distance of all pairs of vertices. 

The {\it Degree/Diameter problem} is the problem of finding the graph with the maximum vertices for given degree $\Delta$ and diameter $D$. 
The order of a graph with degree $\Delta$ ($\Delta > 2$) of diameter D is easily seen to be bounded by 
\[ 1 + \Delta \sum_{k=1}^{D-1} (\Delta - 1)^k\]
which is called {\it Moore bound}.
Moore bound is the general upper bound. On the other hand, already known lower bounds of small degree and small diameter are availabe at \url{http://combinatoricswiki.org}. 
Especially for case of $D=2$ there exists the general construction called the {\it Brown's construction}.
Given the finite field $F_q$ where q is a power of prime, we construct the graph $\B(F_q)$ whose vertices are lines in $F_q^3$ and two lines are adjacent if and only if they are orthogonal. 
We call it the Brown's graph.
The order of $B(F_q)$ is $q^2+q+1$ and the degree of it is $q+1$. The diameter of it is 2 because $B(F_q)$ includes many triangles. 
Any lines are symmetric in $F_q$, so $B(F_q)$ is regular. However it is not simple because of including some loops. 
Removing any loops from $B(F_q)$, we get the simple graph whose degree of vertices are $q+1$ or $q$. 
In this paper, we generalize the Brown's construction, in which we replace a field with a ring, and search new records of the Degree/Diameter Problem.

%%% Hmm %%%
\iffalse

The {\it tensor product} of two graphs $G_1=(V_1,E_1),G_2=(V_2,E_2)$ is a graph $G_1 \otimes G_2$ such that the vertex set is a cartesian product of $V_1$ and $V_2$ and two vertices $(v_1,v_2) \sim (w_1,w_2)$ if and only if $v_1 \sim v_2$ and $w_1 \sim w_2$. If $G_1, G_2$ are regular, the tensor product $G_1 \times G_2$ is also regular. 

\fi
%%% --- %%%

Let $R$ be a ring with unity. 
$R^*$ denotes the set of invertible elements of $R$.
$R^3$ is naturally seen as R-module. 
The addition and $R$-action are defined by coordinate-wise.
The {\it inner product} $\cdot: R^3 \times R^3 \Rightarrow R$ is defined as follows
\[ (v_1,v_2,v_3) \cdot (w_1,w_2,w_3) = v_1 w_1 + v_2 w_2 + v_3 w_3 \]
${\bm v}, {\bm w}$ are {\it orthogonal} if and only if the inner product vanishes, namely ${\bm v} \cdot {\bm w} = 0$.
The {\it cross product} $\times: R^3 \times R^3 \Rightarrow R$ is defined as follows
\[ (v_1,v_2,v_3) \cdot (w_1,w_2,w_3) = ( v_2 w_3 - v_3 w_2, v_3 w_1 - v_1 w_3, v_1 w_2 - v_2 w_1 ) \]
A {\it domain} $D$ is a ring without zero divisors.
A {\it Euclidean domain} is a domain $E$ with a function called degree $d: E \setminus \{0\} \Rightarrow \N$ such that for all non-zero $a,b \in E$ there exists $q,r \in E, a = q b + r$ where $d(r) < d(b)$.
The ring of integers $\Z$ is a example of the Euclidean domains.


\section{Generalized Brown's Construction}
\begin{Def}
Let $(R,+,0,*,1)$ be a ring with unit. The vertex set $V$ of the generalized Brown's graph $\B(R)$ is \[ V = ( R^3 \setminus \{\bm v | \exists r \in R, r \cdot {\bm v} = {\bm 0} \} ) / \sim\]
where $\bm v \sim \bm w$ if and only if there exists $k \in R^*$ such that $k \cdot {\bm v} = {\bm w}$. The two vertices $[\bm v],[\bm w]$ are adjacent if and only if ${\bm v} \cdot {\bm w} = 0$.
\end{Def}

The adjacency of the definition above is well-defined because the orthogonality does not depends on the selection of representitives. We call the above construction which gives a graph from a ring the {\it generalized Brown's construction}. It is clear that the new construction coincides old one when the ring is a field.

%%% Hmmm %%%
\iffalse

\begin{Lem}
Let $R_1, R_2$ be rings with unit. 
\[ \B(R_1 \times R_2) \simeq \B(R_1) \otimes \B(R_2) \]
\end{Lem}

\begin{Cor}
The followig equations hold.
\begin{enumerate}
\item $ | \B(R_1 \times R_2) | = | \B(R_1) | \times | \B(R_2) | $
\item $ \Delta(\B(R_1 \times R_2)) = \Delta(\B(R_1)) \times \Delta( \B(R_2) )$
\end{enumerate}
\end{Cor}

\fi 
%%% --- %%%

\begin{Lem}\label{Lem:regular}
Let $E$ be a Euclidean domain and $u$ be a prime element in $E$. If $E/(u^k)$ is a finite ring, then $\B(E/(u^k))$ is a regular graph, where $(u^k)$ is a principal ideal generated by $u^k$. Note that $\B(E/(u^k))$ is not simple.
\end{Lem}

\begin{proof}
It is clear that the degrees of vertices represented by $(1,0,0), (0,1,0), (0,0,1)$ are same.
Let ${\bm v} = ([a],[b],[c])$ be a representitive of any vertex where $a,b,c \in E$. 
If any element of $[a],[b],[c]$ is not invertible in $E/(u^k)$, 
there exists natural numbers $l,m,n < k$ and some elements $a',b',c'$ in $E$ such that $a=u^l a', b=u^m b', c=u^n c'$. ${\bm v}$ is not a representitive of vertices because of $[u^{min(l,m,n)}] \cdot {\bm v} = ([0],[0],[0])$. 
Therefore, one element of $[a],[b],[c]$ is at leatst invertible.
If $[a]$ is invertible, there exists one-to-one correspondence $\overline{U}: N([(1,0,0)])  \rightarrow N([{\bm v}])$ such that for all $[{\bm w}] \in N([1,0,0])$, $\overline{U}([\bm w] = [ {}^t\!U^{-1} {\bm w} ]$ where U is an invertible matrix defined as follows
\[
 U = \left(
 \begin{matrix}
  [a] & 0 & 0 \\
  [b] & 1 & 0 \\
  [c] & 0 & 1
 \end{matrix} \right)
\]
Therefore, $\delta([\bm v]) = \delta([(1,0,0)])$. In the same way, if $[b]/[c]$ is invertible, then $\delta([\bm v]) = \delta([(0,1,0)])/\delta([(0,0,1)])$. 

\end{proof}

\begin{Lem}\label{Lem:diameter}
Let $E$ be a Euclidean domain and $I$ be an ideal of $E$. The diameter of $\B(E/I)$ is 2.
\end{Lem}

\begin{proof}
For any two distinct vertices $[\bm v]$ and $[\bm w]$, consider the cross product $\bm v \times \bm w$. If $\bm v \times \bm w = \bm 0$, $v_i \cdot \bm w = w_i \cdot \bm v$ for $i=1,2,3$. 
For any vertex $[\bm v]$, the cordinate triple $(v_1,v_2,v_3)$ of ${\bm v}$ are coprime, then there exists $a,b,c \in E/I$ such that $a v_1 + b v_2+ c v_3 = 1$.
\[ \bm v = 1 \cdot \bm v = ( a v_1 + b v_2+ c v_3 ) \bm v = ( a w_1 + b w_2 + c w_3 ) \bm w\]
It is a contradiction then $\bm v \times \bm w \neq \bm 0$. 
If $\bm v \times \bm w$ is a representitive of vertex, $[\bm v \times \bm w]$ is adjacent to $[\bm v]$ and $[\bm w]$.
If $\bm v \times \bm w$ is not a representitive of vertex, there exist $k \in E/I$ and $\bm u \in (E/I)^3$ such that $\bm v \times \bm w = k \cdot \bm u$ and $\bm u$ is a representitive of vertex. $[\bm u]$ is adjacent to $[\bm v]$ and $[\bm w]$.
\end{proof}

\begin{Th}
The following equations hold.
\begin{enumerate}
\item $ |\B(\Z_{p^k})| = p^{2k}+p^{2k-1}+p^{2k-2} $
\item $ \Delta(\B(\Z_{p^k})) = p^k + p^{k-1} $
\item $ D(\Z_{p^k}) = 2$
\end{enumerate}
\end{Th}

\begin{proof}
It is straightforward to show the formula of the order of $\B(\Z_{p^k})$.

\begin{eqnarray*}
|\B(\Z_{p^k})| & = & \frac{|\Z_{p^k}|^3 - |\{ mp | 0 \leq m < k \}|^3}{|\Z_{p^k}|-|\{ mp | 0 \leq m < k \}|} \\ 
& = & \frac{(p^k)^3 - (p^{k-1})^3}{p^k-p^{k-1}} = p^{2k}+p^{2k-1}+p^{2k-2}
\end{eqnarray*}

It is only enough to show that the degree of the vertex represented by $(1,0,0)$ satisfy the formula of the degree of $\B(\Z_{p^k})$. 
It is clear that $\Z_p^{k}$ satisfies the assumption of Lemma \ref{Lem:regular}, then $\B(\Z_p^{k})$ is a regular graph. 
\begin{eqnarray*}
\Delta(\B(\Z_p^k)) & = & \delta([(1,0,0)]) = \frac{|\Z_{p^k}|^2 - |\{ mp | 0 \leq m < k \}|^2 }{|\Z_{p^k}|-|\{ mp | 0 \leq m < k \}|} \\
& = & \frac{(p^k)^2 - (p^{k-1})^2}{p^k-p^{k-1}} = p^k + p^{k-1}
\end{eqnarray*}

Applying Lemma \ref{Lem:diameter}, we get $D(\B(\Z_p^k)) = 2$

\end{proof}

We search new graphs 

\begin{Th}
\end{Th}

\section{Acknowledgement}
Thank you!!!

\end{document}